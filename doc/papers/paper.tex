%&pdflatex
%
%  Comparison of Numerical Models for Coastal Dynamics
%
%  Authors:
%
\documentclass[]{article}

\usepackage{graphicx}

% Use utf-8 encoding for foreign characters
\usepackage[utf8]{inputenc}

% Multipart figures
% \usepackage{subcaption}

% More symbols
\usepackage{amsmath}
\usepackage{amssymb}
\usepackage{latexsym}

% URL and linking help
\usepackage{hyperref}

% Package for including code in the document
\usepackage{listings}
% \DeclareCaptionFont{white}{\color{white}}
% \DeclareCaptionFormat{listing}{%
%   \parbox{\textwidth}{\colorbox{gray}{\parbox{\textwidth}{#1#2#3}}\vskip-4pt}}
% \captionsetup[lstlisting]{format=listing,labelfont=white,textfont=white}
% \lstset{frame=lrb,xleftmargin=\fboxsep,xrightmargin=-\fboxsep,numbers=left,basicstyle=\ttfamily\small,language=python,columns=fixed,commentstyle=\em\color[rgb]{0.133,0.545,0.133}}

% This is now the recommended way for checking for PDFLaTeX:
\usepackage{ifpdf}

% Add line numbers
\usepackage[mathlines]{lineno}

% Need to use this package to handle the spacing issues in LaTeX after a command
\usepackage{xspace}

% Markup
\usepackage{color}
\newcommand{\comment}[1]{\color{blue} #1}
\newcommand{\alert}[1]{\textbf{\color{red} #1}}

\graphicspath{{./figures/}}



% Useful commands
\include{macros}

\begin{document}

\ifpdf
\DeclareGraphicsExtensions{.pdf, .png, .jpg, .tif}
\else
\DeclareGraphicsExtensions{.png, .jpg, .tif, .eps}
\fi

\title{Comparison of Numerical Models for Coastal Dynamics}

\author{Kyle T. Mandli\thanks{
            Columbia University} \and
        Nishant Panda\thanks{
            Colorado State University}
        }

\maketitle

\begin{abstract}
    Put really awesome abstract here.
\end{abstract}

\section{Introduction}

\subsection{Selection Criteria}
\begin{itemize}
    \item Limit models to those that have the following properties:
    \begin{itemize}
        \item Well-balanced
        \item Handles inundation
        \item 2D or at the most semi-3D (layers) - this may troublesome to differentiate as most hydrostatic ocean codes are really layered 3D models
    \end{itemize}
\end{itemize}

\begin{tabular}{c|ccccc}
\textbf{Model} & Well-balanced & Inundation & 2D & Non-Hydrostatic & ~ \\
\hline
SWE                 & X & X & ~ & ~ & ~ \\
Green-Naghdi        & ~ & ~ & ~ & ~ & ~ \\
Boussinesq          & ~ & ~ & ~ & ~ & ~ \\
Full 3D Hydrostatic & ~ & ~ & ~ & ~ & ~
\end{tabular}

\section{Benchmarks}

\begin{itemize}
    \item \url{https://github.com/rjleveque/nthmp-benchmark-problems}
    \item \url{https://github.com/rjleveque/geoclaw-group/tree/master/benchmarks}
    \item \url{https://github.com/rjleveque/tsunami_benchmarks}
    \item \url{http://depts.washington.edu/clawpack/geoclaw/benchmarks/nthmp_currents_2015/}
\end{itemize}

\begin{tabular}{c|cccc}
\textbf{Dimensionality} & \textbf{Benchmark} & Analytical & Experimental & Field \\
\hline \hline
1D & Solitary Wave on a Simple Beach    & X          & X & ~ \\
~  & Solitary Wave on a Composite Beach & X (linear) & X & ~ \\
\hline
2D & 3D Underwater Landslide Tsunami      & ~ & X & ~ \\
~  & Solitary Waves on a Conical Island   & ~ & X & ~ \\
~  & Monai Lab Runup Onto Complex Beach   & ~ & X & ~ \\
~  & Okushiri                             & ~ & ~ & X \\
~  & Submarine Landslides (Zhou and Teng) & ~ & X & ~ 
\end{tabular}

\end{document}
