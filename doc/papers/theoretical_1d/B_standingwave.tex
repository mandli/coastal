%!TEX root = paper.tex
\subsection{Linear standing wave}
The linear standing wave test illustrates the different linear dispersion relations and the ability of the numerical method to represent them in an accurate manner. 
Both, the linearized systems of the hydrostatic and dispersive have an analytic standing wave solution
\begin{align}
\xi(\bx,t)&=-a\sin\left(\kappa x \right)\cos\left(\kappa ct\right), \\
 u(\bx,t) &=a \frac{c}{d}\cos\left(\kappa x\right)\sin\left(\kappa ct\right), \\
 v(\bx,t) &=0, \qquad \forall \ \bx=(x, y)^T \in \Omega, \forall t \in \mathbb{R},
\end{align}
where the phase velocity $c=\frac{\omega_{\text{nh,}\fnh}}{\kappa}$ is chosen for the dispersive equation set and $c=\csw$ for the hydrostatic equation set, respectively. For the \nh\ model, the analytic vertical velocity has to be defined as
\[
w(\bx,t)=-\frac{d}{2}\partial_x u=\frac{1}{2}ac\kappa\sin\left(\kappa x \right)\sin\left(\kappa ct\right),
\]
which is derived directly from \eqref{eq:nh_closure}.
In different model runs, we vary the depth $d$ while keeping the wave length $\lambda=\frac{2\pi}{\kappa}=20$ m and the amplitude $a=0.01$m constant to get ratios for $\frac{d}{\lambda}$ between $0.05$ and $1.0$, and ratios for $\frac{a}{d}$ between $0.01$ and $0.005$, respectively.
We impose double periodic boundary conditions on a grid length of one wave length. The simulation time is chosen long enough to measure one wave period. 

\subsubsection{Results of \nh\ model}
The resulting normalized phase velocities for the shallow water model and the \nh\ equation set with either the linear or the quadratic vertical pressure are displayed in figure \ref{fig:nh_standingwave}. Furthermore, they are compared to their analytical reference phase velocities and the full reference phase velocity as derived in section \ref{sec:dispersion}. 
In each case, the numerical dispersion relation matches the corresponding analytical one precisely.
% The coincidence between numerical and analytical phase velocities is very good. 
In a close neighborhood of the shallow water assumption (i.e., in the limit $\frac{d}{\lambda} \rightarrow 0$), the quadratic vertical pressure profile gives a better phase velocity compared to the full reference solution than the linear profile, as expected from series expansions around this state used in \Bt\ models.
% \svnote{Can we have an additional detail plot of this neighborhood?} \ajnote{Then only thinner lines would help. I could also provide the whole image with different plotting style (thinner lines and ) One could also use absolute errors between numerical values and reference phase velocities. }\jbnote{What are you expecting to see?}\svnote{I expect to see what we describe in the text, i.e., that the quadratic vertical pressure profile gives a better phase velocity than the linear one near $d/\lambda=0$, see added figure.}
However, for ratios $\frac{d}{\lambda} > 0.25$ approximately, the linear profile matches better.

\begin{figure}[htbp]
        \includegraphics*[width=0.95\textwidth]{nh_standingwave.eps}
        \caption{Standing wave: Comparison of simulated hydrostatic and \nh\ phase velocities with analytic reference values for all simulations (left) and a zoom onto the close neighborhood of the long wave limit (right)}
        \label{fig:nh_standingwave}
\end{figure}

